\documentclass[../Problems]{subfiles}
\begin{document}
\begin{center}
	\textbf{\large{How to write a program? (5Cs)}}
\end{center}
\begin{itemize}
\item Carefully go through the problem statement
\item Check your understanding of problem using solved examples and practice testcases
\item Compose the programming approach on paper
\item Consolidate your approach by verifying its correctness on testcases by doing dry runs
\item Code it up (finally!)
\end{itemize}
\begin{center}
	\textbf{\large{Good Programming Practices}}
\end{center}
\begin{itemize}
\item 
Write documentation clearly explaining 
	\begin{itemize}
	\item what the program does,
	\item how to use it,
	\item what quantities it takes as input, and
	\item what quantities it returns as output.
	\end{itemize}
\item
Use appropriate variable/function names.
\item 
Extensive internal comments explaining how the program works.
\item 
Complete error handling with informative error messages.\\
For example, if \verb!a = b = 0!, then the \verb!gcd(a, b)! routine should return the error message \\``\verb!gcd(0,0)! is undefined'' instead of going into an infinite loop or returning a ``division by zero'' error.
\end{itemize}
\begin{center}
	\textbf{\large{Tips}}
\end{center}
\begin{itemize}
\item Some data types that you should keep in mind are:
	\begin{itemize}
	\item \verb!bool!
	\item \verb!char!
	\item \verb!short int, int, long int, long long int! and their \verb!unsigned! counterparts
	\item \verb!float, double, long double!
	\end{itemize}
\item Choose appropriate variable data types according to constraints. Example, if a variable is always an integer then it should be assigned an \verb!int! data type.
\item Whenever possible prefer integer data types over floating point data types which aren't accurate due to floating point errors. Some problems that look like they will need floating point numbers but are solvable using integers are \hyperref[pp:triangletypes]{Triangle Types}, \hyperref[pp:friendlypair]{Friendly Pair} and \hyperref[pp:newtoninterpolation]{Newton Interpolation}.
\item Use \href{https://www.geeksforgeeks.org/type-conversion-in-c/}{type conversion} to your advantage to
	\begin{itemize}
	\item make your program unambiguous.
	\item compute expressions containing variables of different data types.
	\end{itemize}
\item Find more tips at \url{https://paramrathour.github.io/CS101/tips}
\end{itemize}
\begin{center}
	\textbf{\large{Get comfortable with Dry Runs}}
\end{center}
The most important step in debugging
\begin{itemize}	
\item Select a testcase
\item Manually go through the code to trace the value of variables
\item Check if the values of variables matches with their expected values
\begin{itemize}
	\item If they do not match for any variable at any time then your program is incorrect, consider debugging/rewriting it
	\item If they match for all variables at all times, Hurray! your program is correct for the current testcases!
\end{itemize}
\item Now repeat the procedure for a different testcase :)
\end{itemize}
\end{document}