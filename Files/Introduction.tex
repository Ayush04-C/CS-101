\begin{center}
	\textbf{\large{Disclaimer}}
\end{center}
These are \textbf{optional} problems.
As these problems are pretty involving, my advice to you would be to first solve exercises given in slides, lab optional questions and get comfortable with the course content.\\
I have created these problems such that you will learn something new from each problem. Each section builds on the next; so, try to solve the problems only using the \textbf{topics mentioned in that section and previous sections}. They will suffice to solve these problems. Don't forget to look at the \href{https://github.com/paramrathour/CS-101/tree/main/Starter%20Codes}{\textbf{starter code}} (it will be in blue) for each problem which takes care of input and output behaviours (and sometimes provides hints). I have also prepared \textbf{model solutions} for each problem, they are available on request. Some interesting solutions that students have sent to me are available \href{https://paramrathour.github.io/CS101/Solutions}{here}. Feel free to share your programs too at \href{mailto:paramrathour3435@gmail.com}{paramrathour3435@gmail.com}. You can always find the latest version of this problem set at the webpage mentioned in title.
\begin{center}
	\textbf{\large{Acknowledgements}}
\end{center}
Many thanks to \href{https://www.youtube.com/user/numberphile}{Numberphile}, \href{https://www.youtube.com/channel/UCYO_jab_esuFRV4b17AJtAw}{3Blue1Brown}, \href{https://www.youtube.com/channel/UC1_uAIS3r8Vu6JjXWvastJg}{Mathologer}, \href{https://www.youtube.com/channel/UCs4aHmggTfFrpkPcWSaBN9g}{PBS Infinite Series}, \href{https://www.youtube.com/user/1veritasium}{Veritasium} and countless other YouTube channels for developing my love for mathematics and their \emph{Fun Videos} further inspiring me to create these problems. Also thank you \href{https://en.wikipedia.org/wiki/Main_Page}{Wikipedia} and \href{http://oeis.org/}{The On-Line Encyclopedia of Integer Sequences} for freely providing their vast resources and detailed information about concepts which helped me frame these problems. Many numbers, phrases, equations and graphics are directly taken from there and modified as per my wish. I would also like to thank \href{https://projecteuler.net/}{Project Euler}, \href{https://cses.fi/problemset}{CSES}, \href{https://codeforces.com/}{Codeforces} and many other online programming practice communities which motivated me to further pursue programming and create problems. I faced lots of \TeX nical issues while setting up this documen and I thank \href{tex.stackexchange.com}{\TeX - \LaTeX Stack Exchange} community for their support and many thanks to \href{https://latexdraw.com/tikz-cover-pages-gallery/}{\LaTeX Draw} for their stylish cover page. Thanks to the CS101 professors, my fellow TAs, tutees, and others for their valuable suggestions on improving these problems. And, lastly thanks to you, reader; These problems are the result of my hard work over the years. I hope they help you in some way or the other and you enjoy solving them :).
\begin{center}
	\textbf{\large{Simplecpp Graphics}}
\end{center}
Also we will be using \href{https://www.cse.iitb.ac.in/~ranade/simplecpp/}{Simplecpp} for initial problem sets (till \ref{sec:sequences}).
Why? because \href{https://www.cse.iitb.ac.in/~ranade/iticse16.pdf}{Introductory Programming: Let Us Cut through the Clutter!} The course book is \href{https://www.cse.iitb.ac.in/~ranade/book.html}{An Introduction to Programming through C++ by Abhiram G. Ranade}.\\
Apart from C++, Simplecpp graphics are an interesting approach to introductory programming. Check out \href{https://en.wikipedia.org/wiki/Turtle_graphics}{Turtle Graphics -- Wikipedia} and \href{https://www.cse.iitb.ac.in/~ranade/simplecpp/gallery.html}{Simplecpp Gallery} for some fascinating examples. Graphics problems in this problem set are -- \hyperref[pp:starspiral]{Star Spiral}, \hyperref[pp:peace]{Peace}, \hyperref[pp:timestable]{Modular Times Table}, \hyperref[pp:regularstarpolygon]{Regular Star Polygon}, \hyperref[pp:hilbertcurve]{Hilbert Curve}, \hyperref[pp:thuemorsesequence]{Thue-Morse Sequence}, \hyperref[pp:recamanssequence]{Recaman's Sequence}, \hyperref[pp:fareysequence]{Farey Sequence}, \hyperref[pp:dragoncurve]{Dragon Curve}, \hyperref[pp:sierpinskicurve]{Sierpi\'nski Arrowhead Curve}, \hyperref[pp:sierpinskitriangle]{Sierpi\'nski Triangle} and \hyperref[pp:barnsleyfern]{Barnsley's Fern}. %(some are yet to be added).

Here are additional chapters of the book on Simplecpp graphics demonstrating its power.\\
(It is just a list, you are not expected to understand/study things, CS101 is for a reason :P)
%(But you will definitely understand all this after this course)
\begin{description}
	\item[Chapter 1] Turtle graphics
	\item[Chapter 5] Coordinate based graphics, shapes besides turtles
	\item[Chapter 15.2.3] Polygons
	\item[Chapter 19] Gravitational simulation
	\item[Chapter 20] Events, Frames, Snake game
	\item[Chapter 24.2] Layout of math formulae
	\item[Chapter 26] Composite class
	\item[Chapter 28] Airport simulation
\end{description}