\subsection{Pisano Period}
The Fibonacci numbers are the numbers in the integer sequence defined by the following recurrence relation
\begin{equation}
	\begin{aligned}
		F_0 &= 0\\
		F_1 &= 1 \\
		F_n &= F_{n-1} + F_{n-2}\quad n \in \mathbb{Z}\quad\text{(Yes! They can be extended to negative numbers)}
	\end{aligned}
\end{equation}
For any integer $n$, the sequence of Fibonacci numbers $F_i \ \%\ n$ is periodic.

The Pisano period, denoted $\pi(n)$, is the length of the period of this sequence.

For example, the sequence of Fibonacci numbers modulo 3 begins:
\begin{equation*}
	0, 1, 1, 2, 0, 2, 2, 1, 0, 1, 1, 2, 0, 2, 2, 1, 0, 1, 1, 2, 0, 2, 2, 1, 0,\ldots\text{(\href{https://oeis.org/A082115}{A082115})}
\end{equation*}
This sequence has period 8, so $\pi(3) = 8$.

Basically, the remainders repeat when these numbers are divided by $n$. You have to find this period.

\textbf{Problem Statement:}\\
Find Pisano period of $t$ numbers $n_1,n_2,\ldots,n_t$
\begin{testcases}
	{$t$ \hfill(number of test cases, an integer)\\
	$n_1\ n_2\ \ldots\ n_t$ \hfill($t$ space seperated numbers for each testcase)}
	{$\pi(n_i)$ \hfill(each test case space seperated)}
	{$1 < n_i \leq 1000$}
	{17\\2 3 5 8 13 21 34 55 89 144 233 987 30 50 98 750 1000}
	{3\quad8\quad20\quad12\quad28\quad16\quad36\quad20\quad44\quad24\quad52\quad32\quad120\quad300\quad336\quad3000\quad1500}
	{https://github.com/paramrathour/CS-101/tree/main/Starter Codes/Pisano Period.cpp}
\end{testcases}
\begin{funvideo}
\href{https://youtu.be/Nu-lW-Ifyec}{Fibonacci Mystery -- Numberphile}
\end{funvideo}