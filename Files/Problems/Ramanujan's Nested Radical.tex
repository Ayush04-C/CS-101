\documentclass[../../Problems]{subfiles}
\begin{document}
\subsection{Ramanujan's Nested Radical}{\label{pp:ramanujanradical}}
\begin{equation}
{ r={\sqrt{1+2{\sqrt{1+3{\sqrt{1+4{\sqrt{1+\cdots }} }}}}}} = \lim_{n\rightarrow\infty} {\sqrt {1+2{\sqrt {1+3{\sqrt {\cdots \sqrt{1+ n}}}}}}}}
\end{equation}
Let's define $r_n$ as $n$-th iteration of this infinite nested radical as below
\begin{equation*}
{ r_n={\sqrt {1+2{\sqrt {1+3{\sqrt {\cdots \sqrt{1+ n}}}}}}}}
\end{equation*}
\textbf{Problem Statement:}\\
Calculate $r_n$ for all test cases accurate till 10 decimal places. See Starter code (below) for more details.
\begin{testcases}
	{$t$ \hfill(number of test cases, an integer)\\$n_1\ n_2\ \ldots\ n_t$ \hfill($t$ space seperated integers for each testcase)}
	{$r_{n_i}$ \hfill(each test case on a newline, accurate till 10 decimal places)}
	{$2 \leq n_i \leq 100$}
	{8\\2 3 5 10 20 30 50 100}
	{1.7320508076\\2.2360679775\\2.7550532613\\2.9899203606\\2.9999878806\\2.9999999868\\3.0000000000\\3.0000000000}
	{https://github.com/paramrathour/CS-101/tree/main/Starter Codes/Ramanujan's Nested Radical.cpp}
\end{testcases}
\begin{funvideo}
\href{https://youtu.be/HMGZVKwYNfk}{Ramanujan: Knowing The Man Who Knew Infinity -- singingbanana}\\
\href{https://youtu.be/leFep9yt3JY}{Ramanujan's infinite root and its crazy cousins -- Mathologer}
\end{funvideo}
\end{document}