\subsection{Wallis Product}{\label{pp:wallis}}
$\pi/2$ is given by below infinite product formula. It is the ratio of product of even squares and odd squares
\begin{equation}
\frac{\pi}{2} = {\frac {2}{1}}\cdot {\frac {2}{3}}\cdot {\frac {4}{3}}\cdot {\frac {4}{5}}\cdot {\frac {6}{5}}\cdot {\frac {6}{7}}\cdots= \prod _{i=1}^{\infty}\left({\frac {2i}{2i-1}}\cdot {\frac {2i}{2i+1}}\right)
\end{equation}
Let's define $\pi_n$ as $n$-th iteration of this infinite product as below
\begin{equation*}
\frac{\pi_n}{2} = {\frac {2}{1}}\cdot {\frac {2}{3}}\cdot {\frac {4}{3}}\cdot {\frac {4}{5}}\cdot {\frac {6}{5}}\cdot {\frac {6}{7}}\cdots{\frac {2n}{2n-1}}\cdot {\frac {2n}{2n+1}} = \prod _{i=1}^{n}\left({\frac {2i}{2i-1}}\cdot {\frac {2i}{2i+1}}\right)
\end{equation*}
\textbf{Problem Statement:}\\
Calculate $\pi_n$ for all test cases accurate till 10 decimal places. See Starter code (below) for more details.
\begin{testcases}
	{$t$ \hfill(number of test cases, an integer)\\$n_1\ n_2\ \ldots\ n_t$ \hfill($t$ space seperated integers for each testcase)}
	{$\pi_{n_i}$ \hfill(each test case on a newline, accurate till 10 decimal places)}
	{$1 \leq n_i \leq 10^{6}$}
	{11\\1 2 3 5 10 20 30 50 100 1000 1000000}
	{2.6666666667\\2.8444444444\\2.9257142857\\3.0021759546\\3.0677038066\\3.1035169615\\3.1159482859\\3.1260789002\\3.1337874906\\3.1408077460\\3.1415918682}
	{https://github.com/paramrathour/CS-101/tree/main/Starter Codes/Wallis Product.cpp}
\end{testcases}
\begin{funvideo}
\href{https://youtu.be/8GPy_UMV-08}{The Wallis product for pi, proved geometrically -- 3Blue1Brown}\\
\href{https://youtu.be/k9nRlMDbefc}{The World's Most Beautiful Formula For Pi -- BriTheMathGuy}
\end{funvideo}