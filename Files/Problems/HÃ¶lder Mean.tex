\documentclass[../../Problems]{subfiles}
\begin{document}
\subsection{H\"older Mean}
H\"older mean is a generalized notion for aggregating sets of numbers.\\
For any non-zero real number $p$ and positive reals $x_1, x_2,\ldots,x_n$, it is defined as
\begin{equation}
{\displaystyle M_{p}(x_{1},\ldots,x_{n})=\left({\frac {1}{n}}\sum _{i=1}^{n}x_{i}^{p}\right)^{\frac {1}{p}}}
\end{equation}
Its special cases are
\begin{equation}
	\begin{aligned}
		p & = -\infty \quad & \rightarrow & \quad M_{-\infty}(x_1,\dots,x_n)       & {}={} & \lim_{p\to-\infty} M_p(x_1,\dots,x_n) = \min \{x_1,\dots,x_n\}                     & \quad\text{(minimum)}\\
		p & = -1 \quad      & \rightarrow & \quad M_{-1}(x_{1},\dots ,x_{n})       & {}={} & {\frac {n}{{\mfrac {1}{x_{1}}}+\dots +{\mfrac {1}{x_{n}}}}}                        & \quad\text{(harmonic mean)}\\
		p & = 0 \quad       & \rightarrow & \quad M_{0}(x_{1},\dots ,x_{n})        & {}={} & \lim _{p\to 0}M_{p}(x_{1},\dots ,x_{n})={\sqrt[{n}]{x_{1}\cdot \dots \cdot x_{n}}} & \quad\text{(geometric mean)}\\
		p & = 1 \quad       & \rightarrow & \quad M_{1}(x_{1},\dots ,x_{n})        & {}={} & {\frac {x_{1}+\dots +x_{n}}{n}}                                                    & \quad\text{(arithmetic mean)}\\
		p & = 2 \quad       & \rightarrow & \quad M_{2}(x_{1},\dots ,x_{n})        & {}={} & {\sqrt {\frac {x_{1}^{2}+\dots +x_{n}^{2}}{n}}}                                    & \quad\text{(root mean square)}\\
		p & = 3 \quad       & \rightarrow & \quad M_{3}(x_{1},\dots ,x_{n})        & {}={} & {\sqrt[{3}]{\frac {x_{1}^{3}+\dots +x_{n}^{3}}{n}}}                                & \quad\text{(cubic mean)}\\
		p & = +\infty \quad & \rightarrow & \quad M_{+\infty }(x_{1},\dots ,x_{n}) & {}={} & \lim _{p\to \infty }M_{p}(x_{1},\dots ,x_{n})=\max\{x_{1},\dots ,x_{n}\}           & \quad\text{(maximum)}
	\end{aligned}
\end{equation}
\textbf{Problem Statement:}\\
Calculate $M_{p}(x_{1},\dots ,x_{n})$ for all special cases ($p=-\infty, -1, 0, 1, 2, 3, \infty$) and accurate till 5 decimal places.
\begin{testcasesMore}
	{$t$ \hfill(number of test cases, an integer)\\
	$n_i\quad x_1\ x_2\ \ldots x_{n_i-1}\ x_{n_i}$ \hfill($n_i+1$ space seperated numbers for each testcase)}
	{$M_{p}(x_{1},\dots ,x_{n})$ for $p=\{-\infty, -1, 0, 1, 2, 3, \infty\}$ \hfill{(each test case on a newline, accurate till 5 decimal places))}}
	{$1 \leq n_i \leq 50$ \hfill{(an integer)}\\
	$0 < x_i \leq 100$ \hfill{(a double)}\\
	Also assume that the calculations are always within the range of double}
	{4\\2\quad1 1\\5\quad 1 2 3 4 5\\13\quad 1 3 6 10 15 21 28 36 45 55 66 78 91\\33\quad 1 3 6 2 7 13 20 12 21 11 22 10 23 9 24 8 25 43 62 42 63 41 18 42 17 43 16 44 15 45 14 46 79}
	{1.00000\quad1.00000\quad1.00000\quad1.00000\quad1.00000\quad1.00000\quad1.00000\\1.00000\quad2.18978\quad2.60517\quad3.00000\quad3.31662\quad3.55689\quad5.00000\\1.00000\quad7.00000\quad19.67642\quad35.00000\quad45.28797\quad52.26138\quad91.00000\\1.00000\quad9.31362\quad17.70339\quad25.66667\quad32.17424\quad37.42452\quad79.00000}
	{https://github.com/paramrathour/CS-101/tree/main/Test Cases/Holder Mean/Input.txt}
	{https://github.com/paramrathour/CS-101/tree/main/Test Cases/Holder Mean/Output.txt}
	{https://github.com/paramrathour/CS-101/tree/main/Starter Codes/Holder Mean.cpp}
\end{testcasesMore}
\begin{funvideo}
\href{https://youtu.be/NbiveCNBOxk}{Does The Average Person Exist? -- Stand-up Maths}
\end{funvideo}
\end{document}